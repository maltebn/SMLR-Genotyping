% Shared preamble for Enhanced-SNP-genotyping-with-SMLR.tex and smlr-supplementary-material.tex
%%%%%%%%%%%%%%%%%%%%%%%%%%%%%% Packages %%%%%%%%%%%%%%%%%%%%%%%%%%%%%%%
% \subfile{filename} is internally the same as the basic TeX function \include{filename},
% the difference being that subfiles can be compiled as standalones inheriting the preamble from
% the main file. Using \include{filename} corresponds to a plain copy-paste of the content of
% filename.tex into the line where the \include is placed. Thus, if there is no need to compile
% filename.tex as a standalone, one might as well just use \input to keep things simple.
\usepackage{subfiles}

%% Defines \sfrac{}{} for slanted fractions:
\usepackage{xfrac}

%% Defines \mathclap{...} for elimination of horizontal whitespace created by e.g. a long subscript under a summation:
\usepackage{mathtools} % note: mathtools loads amsmath

%% The siunitx package provides a set of tools to typeset quantities in a consistent way:
% a useful command is \qty{<number>}{<unit>} which writes the quantity as a product of
% the number and the unit, so the space here is formally showing multiplication, and it
% is smaller than a standard space, which makes it look prettier.
\usepackage{siunitx}

%% Make the standard latex tables look so much better:
\usepackage{array,booktabs}
\usepackage{multirow}
\usepackage{threeparttable}

%% Make landscape pages display as landscape. (Ideal for pages to be read on a screen)
\usepackage{pdflscape}

%% A better approach to resize e.g. tables to text width
%% See: https://tex.stackexchange.com/questions/121155/how-to-adjust-a-table-to-fit-on-page
\usepackage{adjustbox}

%% Extensive control of page headers and footers
\usepackage{fancyhdr}

%% To handle the formatting of special characters in urls, e.g. in the .bib files, use either the hyperref- or url-package
\usepackage{url}



%%%%%%%%%%%%%%%%%%%%%%%%%%%%%% Draft/submission and blinded modes %%%%%%%%%%%%%%
% Switch between draft and submission mode
\newif\ifsubmission
\submissionfalse   % <- Set to \submissiontrue for submission mode

% Switch between blinded and non-blinded mode
\newif\ifblinded
\blindedfalse   % <- Set to \blindedtrue for blinded mode
%%%%%%%%%%%%%%%%%%%%%%%%%%%%%% end of modes %%%%%%%%%%%%%%%%%%%%%%%%%%%%%%%%%%%%


%%%%%%%%%%%%%%%%%%%%%%%%%%%%%% Referencing logic %%%%%%%%%%%%%%%%%%%%%%%%%%%%%%%
% Supplementary reference macro
\newcommand{\suppfig}[2]{
  \ifsubmission
    Supplementary Fig.~S#1
  \else
    Supplementary Fig.~#2
  \fi
}

\newcommand{\supptab}[2]{
  \ifsubmission
    Supplementary Table~S#1
  \else
    Supplementary Table~#2
  \fi
}

\newcommand{\suppeq}[2]{
  \ifsubmission
    %Supplementary Eq.~(S#1)
    (S#1)
  \else
    %Supplementary Eq.~(#2)
    #2 % Assuming that \eqref is used for equation referencing
  \fi
}
%%%%%%%%%%%%%%%%%%%%%%%%%%%%%% end of logic %%%%%%%%%%%%%%%%%%%%%%%%%%%%%%%%%%%%