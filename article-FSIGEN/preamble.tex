% Shared preamble for Enhanced-SNP-Genotyping-with-SMLR.tex and supplementary-material.tex
%%%%%%%%%%%%%%%%%%%%%%%%%%%%%% Packages %%%%%%%%%%%%%%%%%%%%%%%%%%%%%%%
\usepackage{subfiles}

% For cross-referencing between two independent documents (e.g. main article and supplementary material)
% Each document's aux-file contains information about the references used in that document, so by
% using \externaldocument{supplementary_material} we can link to the supplementary material aux file and
% in that way use reference labels created in another document.
\usepackage{xr}

%% Defines \sfrac{}{} for slanted fractions:
\usepackage{xfrac}

%% Defines \mathclap{...} for elimination of horizontal whitespace created by e.g. a long subscript under a summation:
\usepackage{mathtools} % note: mathtools loads amsmath

%% The siunitx package provides a set of tools to typeset quantities in a consistent way:
% a useful command is \qty{<number>}{<unit>} which writes the quantity as a product of
% the number and the unit, so the space here is formally showing multiplication, and it
% is smaller than a standard space, which makes it look prettier.
\usepackage{siunitx}

%% Make the standard latex tables look so much better:
\usepackage{array,booktabs}
\usepackage{multirow}
\usepackage{threeparttable}

%% Make landscape pages display as landscape. (Ideal for pages to be read on a screen)
\usepackage{pdflscape}

%% A better approach to resize e.g. tables to text width
%% See: https://tex.stackexchange.com/questions/121155/how-to-adjust-a-table-to-fit-on-page
\usepackage{adjustbox}

%% Extensive control of page headers and footers
\usepackage{fancyhdr}

%% To handle the formatting of special characters in urls, e.g. in the .bib files, use either the hyperref- or url-package
\usepackage{url}
\usepackage{doi}