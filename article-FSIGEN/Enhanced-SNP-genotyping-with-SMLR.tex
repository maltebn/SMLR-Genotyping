%%%%%%%%%%%%%%%%
% Main article %
%%%%%%%%%%%%%%%%

%% Use the options 1p,twocolumn; 3p; 3p,twocolumn; 5p; or 5p,twocolumn
%% for a journal layout:
\documentclass[preprint,5p,times,11pt]{elsarticle}
\biboptions{sort&compress}

% Shared preamble for Enhanced-SNP-genotyping-with-SMLR.tex and smlr-supplementary-material.tex
%%%%%%%%%%%%%%%%%%%%%%%%%%%%%% Packages %%%%%%%%%%%%%%%%%%%%%%%%%%%%%%%
% \subfile{filename} is internally the same as the basic TeX function \include{filename},
% the difference being that subfiles can be compiled as standalones inheriting the preamble from
% the main file. Using \include{filename} corresponds to a plain copy-paste of the content of
% filename.tex into the line where the \include is placed. Thus, if there is no need to compile
% filename.tex as a standalone, one might as well just use \input to keep things simple.
\usepackage{subfiles}

%% Defines \sfrac{}{} for slanted fractions:
\usepackage{xfrac}

%% Defines \mathclap{...} for elimination of horizontal whitespace created by e.g. a long subscript under a summation:
\usepackage{mathtools} % note: mathtools loads amsmath

%% The siunitx package provides a set of tools to typeset quantities in a consistent way:
% a useful command is \qty{<number>}{<unit>} which writes the quantity as a product of
% the number and the unit, so the space here is formally showing multiplication, and it
% is smaller than a standard space, which makes it look prettier.
\usepackage{siunitx}

%% Make the standard latex tables look so much better:
\usepackage{array,booktabs}
\usepackage{multirow}
\usepackage{threeparttable}

%% Make landscape pages display as landscape. (Ideal for pages to be read on a screen)
\usepackage{pdflscape}

%% A better approach to resize e.g. tables to text width
%% See: https://tex.stackexchange.com/questions/121155/how-to-adjust-a-table-to-fit-on-page
\usepackage{adjustbox}

%% Extensive control of page headers and footers
\usepackage{fancyhdr}

%% To handle the formatting of special characters in urls, e.g. in the .bib files, use either the hyperref- or url-package
\usepackage{url}

%% The lineno packages adds line numbers. Start line numbering with
%% \begin{linenumbers}, end it with \end{linenumbers}. Or switch it on
%% for the whole article with \linenumbers.
%\usepackage{lineno}
%% For linenumbers at every fifth line use
\usepackage[modulo]{lineno}



%%%%%%%%%%%%%%%%%%%%%%%%%%%%%% Draft/submission and blinded modes %%%%%%%%%%%%%%
% Switch between blinded or non-blinded mode (\blindedtrue or \blindedfalse)
% Blinded: Removes author details from frontmatter and replaces other identifying information with placeholders.
%          Adds linenumbers at every fifth line in main text.
% Non-blinded: Prints author and identifying information. No line numbers.
\newif\ifblinded
\blindedfalse   % <- Set to \blindedtrue for blinded mode

% Switch between draft or submission mode (\submissionfalse or \submissiontrue)
% Draft: Gathers main article and supplementary material into one pdf with clickable cross-referencing.
% Submission: Main article and supplementary material should be compiled separately.
%             References from main to main are clickable, but not from main to supplementary.
%             No clickable references in supplementary material.
\newif\ifsubmission
\submissiontrue   % <- Set to \submissionfalse for draft mode
% Note: Submission mode should be toggled on by default as this is the only option that makes sense
%       when compiling the supplementary material as standalone. With submission mode as default, we don't
%       need to toggle anything in 'wrapper-supplementary-smlr.tex'.
%%%%%%%%%%%%%%%%%%%%%%%%%%%%%% end of modes %%%%%%%%%%%%%%%%%%%%%%%%%%%%%%%%%%%%


%%%%%%%%%%%%%%%%%%%%%%%%%%%%%% Referencing logic %%%%%%%%%%%%%%%%%%%%%%%%%%%%%%%
% Supplementary reference macro
\newcommand{\suppfig}[2]{%
  \ifsubmission
    Supplementary Fig.~S#1%
  \else
    Supplementary Fig.~#2%
  \fi
}

\newcommand{\supptab}[2]{%
  \ifsubmission
    Supplementary Table~S#1%
  \else
    Supplementary Table~#2%
  \fi
}

% Assuming that \eqref is used for equation referencing:
\newcommand{\suppeq}[2]{%
  \ifsubmission
    %Supplementary Eq.~(S#1)
    (S#1)%
  \else
    %Supplementary Eq.~(#2)
    #2%
  \fi
}
%%%%%%%%%%%%%%%%%%%%%%%%%%%%%% end of logic %%%%%%%%%%%%%%%%%%%%%%%%%%%%%%%%%%%%
\submissionfalse   % <- Set to \submissiontrue for submission mode
\blindedtrue   % <- Set to \blindedtrue for blinded mode

\usepackage{doi}


%%%%%%%%%%%%%%%%%%%%%%%%%%%%%% Document settings %%%%%%%%%%%%%%%%%%%%%%%%%%%%%%%

% Higher penalty for linebreaking in inline math (default is \relpenalty=500 and \binoppenalty=700):
\relpenalty=9999
\binoppenalty=9999
% Redefining figure naming (to be redefined again in Supplementary material)
\renewcommand{\figurename}{Fig.}

%%%%%%%%%%%%%%%%%%%%%%%%%%%%%% end of settings %%%%%%%%%%%%%%%%%%%%%%%%%%%%%%%%%



\journal{Forensic Science International: Genetics}

\begin{document}

%% Frontmatter
\subfile{smlr-frontmatter}

%% Main text
\subfile{smlr-main-text}



%% The Appendices part is started with the command \appendix;
%% appendix sections are then done as normal sections
%\newpage
%\appendix



%% Bibliography
\bibliographystyle{elsarticle-num} 
\bibliography{references-smlr}



%% Supplementary material
% If submission mode is toggleg on, the supplementary material should NOT be included in
% the same pdf as the main article, but should be generated in one or more separate file(s).
% However, if in draft mode (non-submission mode), the supplementary material is inserted
% into the same pdf (after the main article) for easier editing.
\ifsubmission\else
  \clearpage
  % Note: Use \clearpage and NOT \newpage, since
  % the former flushes floating elements (like figures and tables) while the latter don't.
  % By using \clearpage, floating elements are then forced to appear BEFORE the \clearpage.
  \subfile{Supplementary-material-smlr}
  %%%%%%%%%%%%%%%%%%%%%%%%%%%
% Supplementary material %
%%%%%%%%%%%%%%%%%%%%%%%%%%
% Note: If we try to use the 'subfiles' package to insert the supplementary contents into this file,
%       then the \fancyhead with the "SUPPLEMENTARY MATERIAL" text is NOT printed on page 8 and 9.
%       This seems to be a bug in the 'subfiles' package, because internally '\subfile{}' is using
%       the basic TeX function '\input{}' and when using this function directly, the header prints
%       just fine. If using the 'subfiles' package, the header also prints fine when the subfile
%       with the supplementary contents is compiled as a standalone. Therefore, it seems like a bug
%       that some headers don't render as expected when using '\subfile{}' from a main file.
%       To have everything compile correctly (also in draft mode when the supplementary material is
%       output in the same pdf as the main article), '\input{}' is used for the supplementary material.
\documentclass[preprint,5p,times,11pt]{elsarticle}
\biboptions{sort&compress}

% Shared preamble for Enhanced-SNP-genotyping-with-SMLR.tex and smlr-supplementary-material.tex
%%%%%%%%%%%%%%%%%%%%%%%%%%%%%% Packages %%%%%%%%%%%%%%%%%%%%%%%%%%%%%%%
% \subfile{filename} is internally the same as the basic TeX function \include{filename},
% the difference being that subfiles can be compiled as standalones inheriting the preamble from
% the main file. Using \include{filename} corresponds to a plain copy-paste of the content of
% filename.tex into the line where the \include is placed. Thus, if there is no need to compile
% filename.tex as a standalone, one might as well just use \input to keep things simple.
\usepackage{subfiles}

%% Defines \sfrac{}{} for slanted fractions:
\usepackage{xfrac}

%% Defines \mathclap{...} for elimination of horizontal whitespace created by e.g. a long subscript under a summation:
\usepackage{mathtools} % note: mathtools loads amsmath

%% The siunitx package provides a set of tools to typeset quantities in a consistent way:
% a useful command is \qty{<number>}{<unit>} which writes the quantity as a product of
% the number and the unit, so the space here is formally showing multiplication, and it
% is smaller than a standard space, which makes it look prettier.
\usepackage{siunitx}

%% Make the standard latex tables look so much better:
\usepackage{array,booktabs}
\usepackage{multirow}
\usepackage{threeparttable}

%% Make landscape pages display as landscape. (Ideal for pages to be read on a screen)
\usepackage{pdflscape}

%% A better approach to resize e.g. tables to text width
%% See: https://tex.stackexchange.com/questions/121155/how-to-adjust-a-table-to-fit-on-page
\usepackage{adjustbox}

%% Extensive control of page headers and footers
\usepackage{fancyhdr}

%% To handle the formatting of special characters in urls, e.g. in the .bib files, use either the hyperref- or url-package
\usepackage{url}

%% The lineno packages adds line numbers. Start line numbering with
%% \begin{linenumbers}, end it with \end{linenumbers}. Or switch it on
%% for the whole article with \linenumbers.
%\usepackage{lineno}
%% For linenumbers at every fifth line use
\usepackage[modulo]{lineno}



%%%%%%%%%%%%%%%%%%%%%%%%%%%%%% Draft/submission and blinded modes %%%%%%%%%%%%%%
% Switch between blinded or non-blinded mode (\blindedtrue or \blindedfalse)
% Blinded: Removes author details from frontmatter and replaces other identifying information with placeholders.
%          Adds linenumbers at every fifth line in main text.
% Non-blinded: Prints author and identifying information. No line numbers.
\newif\ifblinded
\blindedfalse   % <- Set to \blindedtrue for blinded mode

% Switch between draft or submission mode (\submissionfalse or \submissiontrue)
% Draft: Gathers main article and supplementary material into one pdf with clickable cross-referencing.
% Submission: Main article and supplementary material should be compiled separately.
%             References from main to main are clickable, but not from main to supplementary.
%             No clickable references in supplementary material.
\newif\ifsubmission
\submissiontrue   % <- Set to \submissionfalse for draft mode
% Note: Submission mode should be toggled on by default as this is the only option that makes sense
%       when compiling the supplementary material as standalone. With submission mode as default, we don't
%       need to toggle anything in 'wrapper-supplementary-smlr.tex'.
%%%%%%%%%%%%%%%%%%%%%%%%%%%%%% end of modes %%%%%%%%%%%%%%%%%%%%%%%%%%%%%%%%%%%%


%%%%%%%%%%%%%%%%%%%%%%%%%%%%%% Referencing logic %%%%%%%%%%%%%%%%%%%%%%%%%%%%%%%
% Supplementary reference macro
\newcommand{\suppfig}[2]{%
  \ifsubmission
    Supplementary Fig.~S#1%
  \else
    Supplementary Fig.~#2%
  \fi
}

\newcommand{\supptab}[2]{%
  \ifsubmission
    Supplementary Table~S#1%
  \else
    Supplementary Table~#2%
  \fi
}

% Assuming that \eqref is used for equation referencing:
\newcommand{\suppeq}[2]{%
  \ifsubmission
    %Supplementary Eq.~(S#1)
    (S#1)%
  \else
    %Supplementary Eq.~(#2)
    #2%
  \fi
}
%%%%%%%%%%%%%%%%%%%%%%%%%%%%%% end of logic %%%%%%%%%%%%%%%%%%%%%%%%%%%%%%%%%%%%
\usepackage{xr}
\externaldocument{Enhanced-SNP-genotyping-with-SMLR} % Link to supplementary material aux file


% Set to \blindedfalse or \blindedtrue for non-blinded or blinded mode:
\blindedfalse
% Blinded: Removes author details from frontmatter and replaces other identifying information with placeholders.
%          Adds linenumbers at every fifth line in main text.
% Non-blinded: Prints author and identifying information. No line numbers.

% We only want to compile the supplementary material as standalone assuming submission mode.
% In the preamble, the submission mode is toggled on by default, so we don't need to do it in this file.

\begin{document}
\input{supplementary-contents-smlr}
\end{document}
\fi

%% Note: Most publishers want the supplementary material and the main article to be uploaded as two separate files.
%% The publishers will then typically only care about typesetting the main article to match their layout while the supplementary material will just be made available as the file(s) the corresponding author has uploaded.
%% Many articles are generated with clickable reference links, e.g., by use of the 'doi'-package, but when the main article and the supplementary material are separate files, links from the one file to the other will be broken and behave weird.
%% This explains why Elsevier forces the authors to manually type references from the main article to the supplementary material: to avoid dead reference links (which will appear annoying and amateurish to readers).
%% The referencing logic defined in the preamble for this document was designed to help the author in the transition from editing versions to review/submission versions, such that the manual typing of references to supplementary figures, tables, equations etc. becomes easier and reduces the risk of mistyped references.

\end{document}